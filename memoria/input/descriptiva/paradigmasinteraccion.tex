\section{Paradigmas de interacción}

Nuestra aplicación implementará una navegación y funcionalidades utilizando tanto gestos como voz:

\subsection{Integración de gestos}
A parte de implementar los gestos básicos presentes en cada aplicación (desplazar el dedo, doble toque, etc$\ldots$). Se desea implementar diversos gestos para poder acceder fácilmente desde cualquier parte de la aplicación, por ejemplo, al dibujar una C se muestra el menú de comedores de la semana.

\subsection{Integración de voz}

Arandita es un bot con el que puedes chatear y hablar, sobre la UGR y sus distintos centros, además de ofrecerte interesantes curiosidades sobre la universidad.

También se quiere implementar la opción de hablar con el bot para solicitar  información sobre los horarios de las facultades, su calendario académico, menú de comedores…

Arandita también permite la grabación de recordatorios para guardar fechas de exámenes, entregas u otros eventos importantes para el estudiante.

Se desea implementar una opción que permita a Arandita describa lo que aparezca en pantalla para los usuarios con discapacidad visual.
