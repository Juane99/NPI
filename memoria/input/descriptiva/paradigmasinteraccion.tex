\section{Paradigmas de interacción}

Nuestra aplicación implementará una navegación y funcionalidades utilizando tanto gestos como voz:

\subsection{Integración de gestos}
Hemos implementado un gesto de barrido lateral con dos dedos que nos permite volver atrás en cualquier punto de la aplicación.
Se ha implementado un gesto que consiste en dibujar una C y muestra el menú de comedores de la semana.
Desplazar el teléfono hacia adelante muestra la información sobre los grados.
Desplazar el teléfono hacia atrás te hace retroceder al desde un fragmento a la página de inicio, o desde la página de inicio al menú de selección de facultad.
Se ha implementado una brújula.

\subsection{Integración de voz}

Arandita (nombre provisional) es un bot con el que puedes chatear y hablar, sobre la UGR y sus distintos centros, además de ofrecerte interesantes curiosidades sobre la universidad.

También se quiere implementar la opción de hablar con el bot para solicitar  información sobre los horarios de las facultades, su calendario académico, menú de comedores…

