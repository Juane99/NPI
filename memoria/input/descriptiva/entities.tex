\section{Entities}

Para la correcta captación de la información proporcionada por el usuario a través de dialogflow hemos creado las siguientes entities:
\begin{itemize}
	\item centros
	\item grados
	\item infogrados
	\item informacion
	\item menus
\end{itemize}

La ventaja que nos proporciona el uso de las entities es que podemos estandarizar la información captada por parte del usuario.

\subsection{centros}
La entidad centros representa los centros sobre los que el bot es capaz de proporcionar información. En nuestro caso son los siguientes:

\begin{itemize}
	\item Facultad de Bellas Artes.
	\item Escuela Técnica Superior de Ingenierías Informática y de Telecomunicación.
	\item Facultad de ciencias
	\item Escuela Técnica Superior de Ingeniería de Caminos, Canales y Puertos.
	\item Facultad de Ciencias de la Educación.
\end{itemize}

\subsection{grados}
La entidad grados representa los grados sobre los que el bot es capaz de proporcionar información. Nosotros hemos optado por añadir:

\begin{itemize}
	\item magisterio
	\item informática
	\item BBAA
	\item Farmacia
	\item telecomunicaciones
	\item matematicas
	\item Arqueologia
	\item fisica
	\item quimica
	\item Derecho
	\item turismo
	\item edificación
\end{itemize}

\subsection{infogrados}
La entidad infogrados representa que información dispone el bot sobre un grado en concreto. En nuestro caso proporcionamos la siguiente info:

\begin{itemize}
	\item Guia docente - Las guias docentes del grado.
	\item facultad - La facultad o centro en el que se imparte el grado.
\end{itemize}

\subsection{informacion}
La entidad informacion representa lo que sabe nuestro bot sobre el centro que se le ha proporcionado. Concretamente puede proporcionar la siguiente información:

\begin{itemize}
	\item Lugar - Donde esta el centro.
	\item Contacto - Cual es su información de contacto.
	\item Decano - Información sobre el decanato o dirección del centro.
	\item Delegacion - Información sobre la delegación de centro.
\end{itemize}

\subsection{menus}
La entidad menus representa los distintos menus de los que dispone la UGR.

\begin{itemize}
	\item Almuerzo
	\item Cena
	\item Ovolacteovegetariano
	\item Vegano
	\item T-celiaco
\end{itemize}
