\section{Gradle}

El fichero Gradle es el fichero de construcción del paquete. Aquí se definirá la forma en la que se construirá el proyecto, desde los plugins con los que trabajamos para desarrollar el proyecto, a que versión de android está dirigido, cuál es la versión de android mínima, con que versión del SDK de android compilaremos y las dependencias necesarias para compilarlo.


En nuestro caso, además del plugin principal de desarrollo en android \texttt{com.android.application} utilizaremos dos plugins para hacer el desarrollo en kotlin más sencillo, \texttt{kotlin-android} y \\ \texttt{kotlin-android-extensions}.

Con respecto a que versión de android está dirigido el proyecto, debido a que todos los integrantes del grupo disponemos de un teléfono con android 10, el proyecto tiene como objetivo la versión 30 del SDK de android, la correspondiente a android 10, también compilaremos con el SDK 30 de android. Aún así hemos decidido que la versión mínima para ejecutar el proyecto sea el SDK 29, es decir, android 9.0, para realizar pruebas en otros dispositivos.



Como dependencias, tendremos las siguientes:

\begin{itemize}
	\item \texttt{org.jetbrains.kotlin}: Para utilizar como lenguaje de programación Kotlin
	\item \texttt{androidx.core}: Dependencia básica de android.
	\item \texttt{androidx.appcompat}: Dependencia básica de android.
	\item \texttt{com.google.android.material}: Interfaz de Google para android.
	\item \texttt{androidx.constraintlayout}: Dependencia básica de android para gestionar los layout.
	\item \texttt{androidx.navigation}: Dependecia para gestionar la navegación haciendo uso de grafos de navegación.
	\item \texttt{androidx.legacy}: Dependencia básica de android.
	\item \texttt{com.google.android.gms}: API de Google para gestionar la geoposición.
	\item \texttt{com.google.maps.android}: API de Google Maps.
\end{itemize}


En nuestro caso, la mayoría de las funcionalidades se pueden utilizar con la base de android, sin embargo algunas cosas concretas como la geoposición, o gestionar un fragment de Google Map.


