
\section{Android Manifest}

En este fichero encontraremos información esencial del proyecto, como el nombre del paquete, los permisos que necesita el proyecto, la información sobre nombre, descripción, icono, etc de nuestra aplicación, entre otra información de interes.

En concreto para nuestro caso el paquete será \texttt{com.npi\_grupo4.guiaestudiantes}.

Y debido a las funcionalidades que hemos implementado, se necesitarán los siguientes permisos:

\begin{itemize}

	\item \texttt{android.permission.ACCESS\_FINE\_LOCATION}
	\item \texttt{android.permission.ACCESS\_COARSE\_LOCATION\_LOCATION}
	\item \texttt{android.permission.ACCESS\_COARSE\_LOCATION}
	\item \texttt{android.permission.INTERNET}
	\item \texttt{android.permission.ACCESS\_NETWORK\_STATE}

\end{itemize}

Los tres primeros elementos para tener acceso a la localización y los dos últimos para acceder a internet y a la red del dispositivo.

Otra información de interes que tenemos es la clave de la API de Google Maps, necesaria para que funcionen los fragments de Google Maps, así como que actividad será la actividad principal, donde comenzará la aplicación.


